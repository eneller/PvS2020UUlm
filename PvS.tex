\documentclass[12pt,a4]{article}

\usepackage{graphicx}
\usepackage{amssymb}
\PassOptionsToPackage{hyphens}{url}\usepackage{hyperref}%allows urls to follow line breaks of text
\usepackage{xcolor}
\usepackage{comment}
\usepackage[german]{babel}
\usepackage{wrapfig} %allows text to wrap images



\usepackage[utf8]{inputenc}%implements vowel mutations
\raggedright %shifts text to the left page border


%\begin{comment}

%sets colors for links
\hypersetup{
	colorlinks=false,
	urlbordercolor=red,
	%urlcolor=red,
	%linkcolor=red,
	%filecolor=red,      
	%urlcolor=red,
	pdftitle={Seminararbeit},
	%bookmarks=true,
	%pdfpagemode=FullScreen,
}
%\end{comment}
%the figure element allows you to insert a logo on the title page
\title{
	\begin{figure}
		\centering
		\includegraphics[height=1.5cm]{logo.png}
	\end{figure}
Programmierung von Systemen 
}

\author{Erik Neller | Dozent: Matthias Tichy}
%\date{18.08.2020}

\begin{document}
	
	\maketitle
	\thispagestyle{empty} %removes page number from title
	\newpage
	
	\tableofcontents
	\newpage
	

	\section{UML-Klassendiagramme}
	\subsection{Klassen und Schnittstellen}
	Methoden und Attribute werden wie in Java in der Form \texttt{\$TYP \$NAME} angegeben, ist die Methode vom Typ \texttt{void} entfällt der Rückgabetyp.
	\begin{itemize}
		\item \texttt{private} wird durch ''-'' symbolisiert
		\item \texttt{package} wird durch ''{\raise.17ex\hbox{$\scriptstyle\sim$}}'' symbolisiert
		\item \texttt{protected} wird durch ''\#'' symbolisiert
		\item \texttt{public} wird durch ''+'' symbolisiert
		\item \texttt{static}  wird \underline{unterstrichen}
	\end{itemize}
	\texttt{\{readonly\}} ist ein UML Attribut das für Konstanten verwendet wird. 
	\newline\newline
	
	Der Block einer Klasse / eines Interfaces besteht aus drei Feldern: 
	\begin{enumerate}
		\item Name mit Modifikatoren
		\item Attribute (Variablen)
		\item Methoden
	\end{enumerate}
	 Modifikatoren für die Klasse können sein: \texttt{\flqq interface\frqq} oder \texttt{\{abstract\}} bzw. der Name in \textit{kursiv}.
	 
	 Die Multiplizität wird genutzt um Listen und Mengen von Objekten anzugeben, aber auch in Relationen zwischen Klassen:
	 \begin{center}
	 	\includegraphics[width=0.5\linewidth]{images/multiplicity}
	 \end{center}
	 \begin{enumerate}
	 	\item Typ
	 	\item Größe begrenzt \{1..x\} oder unendlich [*]
	 	\item Attribute wie \{order\} und \{unique\}
	 \end{enumerate}

	 \subsection{Relationen}

 	\texttt{extends} oder \texttt{implements} wird dargestellt durch einen nicht ausgefüllten Pfeil, wobei die Linie bei gleichen Typen (Interface erbt von Interface, Klasse von Klasse) durchgezogen ist, wenn eine Klasse ein Interface implementiert, gestrichelt.
 	
 	\begin{center}
 		\includegraphics[width=0.5\linewidth]{images/interface1}
 	\end{center}

		
	 
	 \begin{itemize}
	 	\item Abhängigkeit (Dependency): User nutzt Ressource, aber die Ressource ist nicht Teil der User Klasse. Wird die Ressource modifiziert, muss auch der User modifiziert werden. 
	 		\begin{center}
	 		\includegraphics[width=0.5\linewidth]{images/dependency}
	 		\end{center}
 		
 		\item Aggregation: ''ist Teil von'', wird symbolisiert durch einen Pfeil mit leerer Raute
	 	\begin{center}
	 		\includegraphics[width=0.5\linewidth]{images/aggregation}
	 	\end{center}
	 	
	 	\item Komposition ''besteht aus'', wird symbolisiert durch einen Pfeil mit ausgefüllter Raute
	 	\begin{center}
	 		\includegraphics[width=0.5\linewidth]{images/composition}
	 	\end{center}
 	
 		\item Assoziation: Die Klassen sind auf eine beliebige Weise verbunden, nutzen Methoden der anderen aber nicht auf die oben genannten Weisen
 		\begin{center}
 			\includegraphics[width=0.4\linewidth]{images/association}
 		\end{center}
	 \end{itemize}
	
	Kommentare werden mit einer gestrichelten Linie mit dem eigentlichen Objekt verbunden.
	
	
	\section{Version Control Systems}

	\subsection{Grundlegende Funktionen}
		Version Control Systems (VCS) erlauben die Verwaltung von mehreren Teilen und Versionen eines Projekts und damit die Zusammenarbeit von mehreren Teilnehmern.
	\begin{itemize}
		\item Rechteverwaltung (z.B. Entwickler von Front-und Backend, Projektmanagement)
		\item Archivierung in verschiedenen Versionen (einfach anhand der gemachten Änderung, anstatt vollständige Backups zu machen)
		\item Speicherung von Metadaten: Historie von Änderungen mit Datum, Autor, etc.
		\item Backup zur Wiederherstellung lokal gelöschter Daten oder versehentlicher Änderungen
		\item Zentralisierung auf Server
		
	\end{itemize}
	\subsection{Git}
	Das Git System besteht aus vier Teilen, von denen drei durch Git selbst implementiert werden. Jede Änderung durchläuft sie in dieser Reihenfolge: 
	\begin{enumerate}
		\item Der eigentlichen Arbeitsplatz / \textit{Workspace} in dem Änderungen an Dateien vorgenommen werden
		\item Die \textit{staging area}, in der \textit{commits} aus einzelnen Änderungen an Dateien feingranular (bis zu einzelnen Zeilen) zusammengesetzt werden
		\item Das \textit{lokalen Repository}
		\item Das \textit{remote Repository} auf einem Server, beispielsweise GitHub oder GitLab
	\end{enumerate}
	Auf den einzelnen Bereichen existieren verschiedene Befehle. Für die staging area:
	\begin{itemize}
		\item \texttt{git init} erstellt ein neues Git-Repository im aktuellen Verzeichnis
		\item \texttt{git add} um Dateien der staging area hinzuzufügen
		\item In einer \texttt{.gitignore} Datei können Regeln für Dateien angegeben werden, die generell nicht mit in die staging area aufgenommen werden sollen
		\item \texttt{git status} um aktuell geänderte und getrackte Dateien zu sehen
		\item \texttt{git rm --cached} um Dateien aus der staging area zu entfernen
		
	\end{itemize}
	%\newline
	Für das lokale Repository:
	\begin{itemize}
		\item \texttt{git commit -m \$MESSAGE} um den Inhalt der staging area in das lokale Repository hinzuzufügen
		\item \texttt{git checkout \$COMMIT-HASH} erlaubt das Wiederherstellen vorheriger Zustände von commits
		\item \texttt{git log} zeigt den Verlauf von commits an
		\item \texttt{git remote add \$NAME \$ADDRESS} verknüpft das lokale Repository mit einem remote Repository
	\end{itemize}

	Und für das remote Repository:
	\begin{itemize}
		\item \texttt{git push \$REPOSITORY \$BRANCH} um den lokalen commit auf den Server zu legen
		\item \texttt{git pull} um das lokale Repository mit den Änderungen aus dem remote Repository zu aktualisieren
		\item \texttt{git clone \$REPOSITORY} um das ganze Repository lokal zu speichern
	\end{itemize}


	
	Für verschiedene Themengebiete / Zuständigkeitsbereiche existiert das Konzept von Branches(Verzweigungen), die mit \texttt{git branch \$NAME} erstellt werden können, um anschließend mit \texttt{git checkout \$NAME} in den Branch zu wechseln, oder in Kurzform: \texttt{git checkout -b \$NAME}. Mit \texttt{git merge \$NAME} kann dann ein Branch in den jeweils Aktuellen integriert, oder mit \texttt{git branch -d \$NAME} gelöscht werden. Ist keine automatische Vereinigung der Branches möglich, müssen die angezeigten Dateien manuell geändert und anschließend mit \texttt{git add \$FILE} Alternativ kann der Branch in das Repository aufgenommen werden: \texttt{git push \$REMOTE \$BRANCH}.
	
	
	


	\section{Objektorientierung}
	\subsection{Interfaces}
	Analog zur abstrakten Klasse erlaubt ein Interface keine Instanziierung. Es enthält keine tatsächliche Implementierung, sondern nur Methodenrümpfe und evtl. Konstanten und muss dementsprechend in jeder Klasse mit dem Schlüsselwort \texttt{implements} implementiert werden. Im Gegensatz zu abstrakten Klassen können mehrere Interfaces von einer Klasse implementiert werden.
	
	\subsection{Sichtbarkeit}
	
	\begin{center}
		\includegraphics[width=0.7\linewidth]{images/modifiers}
	\end{center}
	
	
	\subsection{Annotations}
	Verschiedene Standardannotationen, eigene definierbar. Durch Reflection zur Laufzeit auslesbar.
	\begin{itemize}
		\item \texttt{@Override} gibt an dass hier ein Element überschrieben werden soll
		\item \texttt{@Deprecated} gibt eine Warnung aus dass Element veraltet ist
		\item \texttt{@SuppressWarnings()} unterdrückt die angegebene Warnung
		\item \texttt{@Documented} nimmt nachfolgende Annotationen in die JavaDoc mit auf
	\end{itemize}
	JavaDoc:
	\begin{itemize}
		\item \texttt{@author}
		\item \texttt{@version}
		\item \texttt{@param} zur Beschreibung von Methodenparametern
		\item \texttt{@return} beschreibt den Rückgabewert
		\item \texttt{@exception}, \texttt{@throws} Beschreibt Fehlermeldungen, die diese Methode produzieren kann
		\item \texttt{@link} Verknüpfung zu anderem Symbol
	\end{itemize}
	
	\subsection{Enumerations}
	Mit einem \texttt{enum} kann eine Menge von Konstanten definiert werden, entweder auf Klassenlevel oder innerhalb einer Klasse. Sie sind auch selbst eine Klasse, deren Attribute \texttt{public static final} definiert sind. Methoden: 
	\begin{itemize}
		\item \texttt{int ordinal()} gibt die Position in der Liste des Enums zurück
		\item \texttt{String name()} gibt den Namen der Konstanten zurück
		\item \texttt{int valueOf} gibt den zugeordneten Wert der Variablen zurück 
	\end{itemize}
	
	\section{Collections-Generics}
	\subsection{Datenstrukturen}
	\subsubsection{Array}
	Statische Größe, wird mit Länge und Datentyp gespeichert, zB int[5]. Objekt x erhalten: array[x]. Länge: array.length
	\subsubsection{ArrayList}
	Klasse die zur Speicherung Arrays verwendet, diese aber ersetzt wenn die Methoden \texttt{add()} oder \texttt{remove()} aufgerufen werden. Mit \texttt{toArray()} kann der Array erhalten werden, mit \texttt{size()} die Größe, da diese dynamisch ist. Objekt erhalten: arraylist.get(x).
	\subsubsection{Linked List}
	Jedes Listenelement enthält einen Zeiger auf das nächste Element, bei dem letzten Element ist der Zeiger \texttt{NULL}.
	\subsubsection{Double Linked List}
	Wie linked list, nur dass jedes Element zusätzlich einen Zeiger auf das vorherige Element enthält (beim ersten Element \texttt{NULL}).
	\subsubsection{Sorted Tree}
	Setzt die Sortierbarkeit der Objekte voraus: Objekte mit kleinerem Wert sind links des Vaterknotens, mit größerem rechts. Beim Einfügen / Entfernen reorganisieren: rot-schwarz Bäume, B Baum, B+Baum
	\begin{center}
		\includegraphics[width=0.5\linewidth]{images/sortedtree}
	\end{center}
	\subsubsection{HashTable}
	Objekte werden gehasht, also ein Wert aus ihnen berechnet, anhand dessen sie in in eine Tabelle eingeordnet werden.
	\subsubsection{Map}
	Speichert nicht einzelne Werte, sondern Tupel aus (Key, Value), nutzt dann zB eine Liste aus Tupeln oder eine Hashtabelle anhand der Keys.
	
	\subsection{Collections API}
	Stellt effiziente Datenstrukturen für viele Anwendunge bereit, sodass Datenstrukturen und Algorithmen nicht selbst implementiert werden müssen.
	
	\begin{itemize}
		\item \texttt{LinkedList}: add, remove, get, clear, set, indexOf, size, contains
		\item \texttt{TreeSet}: add, remove, contains, size
	\end{itemize}
	
	
	
	
	
	
	
	
	
\end{document}